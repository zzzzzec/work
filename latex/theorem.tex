\documentclass{article} 
\usepackage{amsthm}
\usepackage {mathtools}
\usepackage{xeCJK}
\usepackage{amssymb}
\setCJKmainfont{SimSun}
\usepackage{stackengine}

\newtheorem{theorem}{Theorem1}

\begin{document}

\begin{theorem}
    对于一个有向无环图$G$, 插入一条边$(S_u,S_v)$使得$G$中出现环,记环中的所有节点集合为$C$, 将把$C$中的节点合并为新节点$S_{new}$得到的DAG记为$G_{new}$, 则对于$G_{new}$
    中的任何一个节点$S_i$,记$S_i$的入边集合为$IN_Si$, 出边集合为$OUT_Si$ ,如果$S_i \notin C$, 那么$C \cap IN_Si \neq \emptyset$ 与$C \cap OUT_Si \neq \emptyset$至多有一个成立。 
\end{theorem}

\begin{proof} 
    使用反证法,假设$C \cap IN_Si \neq \emptyset$ 与$C \cap OUT_Si \neq \emptyset$同时成立, 取$S_j \in C \cap IN_Si$, $S_k \in C \cap OUT_Si$, 由于合并后节点$S_j$与节点$S_k$之间存在一条边, 所以在$G_{new}$中$S_j$与$S_k$
    属于同一个强连通分量,则必然存在一条路径$p$,使得 $ S_k \xrightarrow[]{p} S_j $. 此时$G_{new}$中存在环$S_i \xrightarrow[]{} S_k \xrightarrow[]{p} S_j \xrightarrow[]{} S_i$, 与$G_{new}$为有向无环图相矛盾.
    故$C \cap IN_Si \neq \emptyset$ 与$C \cap OUT_Si\neq \emptyset$至多有一个成立。
\end{proof}

\begin{theorem}
   对于一个已经建立完成的索引图$G_I$, 若其中有根据规则3创建的边$(<S_i, L_1>, <S_i, L_2>), L_2 \in L_1$,那么$|L_1| = 1$, 即$L_1$中只含有一个时刻.
\end{theorem}

\begin{proof} 
    有问题!!! \newline
    索引图建立时,构造与$S_i$相关的出入边共有四种情况。
    \begin{center}
    \begin{itemize}
        \item $S_i$作为$S_j$在$t_x$时刻的$N_1$出边, 构造边$(<S_j, t_x>,<S_i,t_x>)$
        \item $S_i$作为$S_j$在$L_2(S_j, S_i)$时刻的$N_2$出边, 构造边$(<S_j, L^+(S_j)>,<S_i, L_2(S_j, S_i)>)$
        \item $S_i$在$t_x$时刻有$N_1$出边$S_k$, 构造边$(<S_i, t_x>,<S_k,t_x>)$
        \item $S_i$在$L_2(S_i, S_k)$时刻有$N_2$出边$S_k$, 构造边$(<S_i, L^+(S_i)>,<S_k, L_2(S_i, S_k)>)$
    \end{itemize}
    \end{center}
    上述情况中, 
    第二种情况构造的边$(<S_j, L^+(S_j)>,<S_i, L_2(S_j, S_i)>)$表示$S_i$在$L_2(S_j, S_i)$时刻必有入边节点$S_j$,
    第四种情况构造的边$(<S_i, L^+(S_i)>,<S_k, L_2(S_i, S_k)>)$表示$S_i$在$L^+(S_i)$时刻必定没有入边。\newline
    因此必然满足$\bigcup \limits_{S_j \in G_S, S_j \neq S_i} L_2(S_j, S_i) \quad \cap \quad L^+(S_i) = \emptyset$, 因此$L_2$只能含有有一个时刻, 即$|L_2| = 1 $.
\end{proof}

\begin{theorem}
    对于一个有向无环图$G$,若加入一条边$(u,v), u \in G, v \in G$使得$G$中产生了强连通分量, 那么必定存在至少一个环$C$, 且$u \in C, v \in C$.
\end{theorem}
\begin{proof} 
    在$G$中,节点$u$和节点$v$的关系一共有三种情况
    \begin{center}
        \begin{itemize}
            \item $u$与$v$相互不可达.
            \item $G$中存在至少一条路$p$, 使得$u \stackrel{p}{\rightsquigarrow} v$, 即$u$可达$v$
            \item $G$中存在至少一条路$q$, 使得$v \stackrel{q}{\rightsquigarrow} u$, 即$v$可达$u$
        \end{itemize}
    \end{center}
    $u \stackrel{p}{\rightsquigarrow} v$和$v \stackrel{q}{\rightsquigarrow} u$不可能同时存在,因为这与$G$是有向无环图相矛盾.
    在第一种情况中, 新加入$(u,v)$最多只能使得$u \rightsquigarrow v$, 不满足$G$中产生强连通分量的条件
    在第二种情况中,新加入$(u,v)$, 由于$u, v$已经可达, 因此加入后仍无法满足$G$中产生强连通分量的条件
    在第三种情况中, 新加入$(u,v)$会产生至少一个环$v \stackrel{q}{\rightsquigarrow} u \to v$.
    综上所述,对于一个有向无环图$G$,若加入一条边$(u,v), u \in G, v \in G$使得$G$中产生了强连通分量, 那么必定存在至少一个环$C$, 且$u \in C, v \in C$.
\end{proof}
 
\end{document}
